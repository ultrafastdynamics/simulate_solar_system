\documentclass{scrartcl}
\usepackage{ngerman}
\usepackage[utf8]{inputenc}
\usepackage{graphicx}
\usepackage{xcolor}
\usepackage{geometry}
\usepackage{listings}
\usepackage{tikz}
\usepackage{hyperref}
\usepackage{enumitem}
%\usepackage[amssymb, textstyle]{SIunits}
\geometry{top=25mm,bottom=25mm,left=25mm,right=25mm}
\pagestyle{empty}
\title{Physik im Weltall -- Computersimulation von Planetenbewegungen}
\date{}
\begin{document}
\begin{center}
    {\huge\textbf{\sffamily Physik im Weltall:\\ Computersimulation von Planetenbewegungen}\\[1cm]}
\end{center}
Diese Simulation erlaubt das Erstellen von Himmelskörpern und die Berechnung ihrer durch Gravitation verbundenen Bewegungen.
Dabei werden die Himmelskörper als Punktmassen modelliert, deren Wechselwirkungen allein durch die Gravitationskraft bestimmt werden.
Ziel dieses Workshops ist es, sowohl das Gravitationsgesetz als auch die numerische Lösung von Problemen (und die Grenzen numerischer Methoden) besser zu verstehen.
Im ersten Teil dieser Anleitung werden die verschiedenen Programmbestandteile erklärt, während im zweiten Teil Arbeitshinweise gegeben werden.
\\[0.5cm]
\begin{center}
        {\Large\textbf{\sffamily Erklärung der Programmbestandteile}}\\[.5cm]
\end{center}
    %\section*{Wichtige Befehle}
        \lstset{basicstyle=\texttt, tabsize=2, morekeywords=Programmname, keywordstyle=\textsc}% backgroundcolor=\color{yellow!15!white}, frame=single}

        \begin{itemize}[leftmargin=*]
            \setlength\itemsep{0em}
            \item \textbf{\color{red}Wichtiger Hinweis:} Zahlen in englischer, wissenschaftlicher Notation: $149,6\cdot10^9 = 149.6\mathrm{e}9$
            \item Im linken Bereich der grafischen Oberfläche kannst du diverse Einstellungen vornehmen und die Geschwindigkeiten der Himmelskörper während der Simulation ablesen.
            \item Im rechten Bereich der grafischen Oberfläche werden dir die Himmelskörper angezeigt, sobald die Simulation gestartet ist.
            \item In der oberen Leiste kannst du die Simulation steuern.
            \item Einstellungen im linken Bereich:
                \begin{itemize}
                    \item Numerik:
                    \begin{itemize}
                        \item \textbf{Timestep:} Zeitschritt in Tagen; Schrittweite, in der die neuen Positionen berechnet werden (z.B.: alle $0,001$ Tage wird eine neue Position berechnet).
                        \item \textbf{Solver:} Art der numerischen Lösungsmethode:
                        \begin{itemize}
                            \item Euler: Euler Verfahren; Simple Methode; Feinerer Zeitschritt notwendig
                            \item RK4: Runge Kutta 4. Ordnung; Komplexere Methode; Größerer Zeitschritt möglich
                        \end{itemize}
                    \end{itemize}
                    \item Grafik:
                    (Diese Einstellungen kannst du auch ändern, wenn die Simulation gerade läuft.)
                    \begin{itemize}
                    \item \textbf{Show object trajectories as circles/lines:} Wenn einer dieser Haken gesetzt ist, bekommst du angezeigt welchen Weg die Himmelskörper zurückgelegt haben. 
                    \item \textbf{Planet image radius:} Hier kannst du einstellen, wie groß die Himmelskörper auf der rechten Seite angezeigt werden. 
                    Da die Himmelskörper als Punktmassen modelliert werden, spielt die Größe keine Rolle und alle Himmelskörper werden gleich groß angezeigt.
                    \item \textbf{Follow object:} Hier kannst du auswählen, ob das Koordinatensystem sich mit einem Objekt bewegen soll.
                    (Mit "`Strg+g"' kannst du die Gitterlinien im Plot entfernen oder wieder hinzufügen.)
                    \end{itemize}
                    \item \textbf{Objects:} Hier kannst du einstellen mit welchen Himmelskörpern gerechnet werden soll. Du kannst aus voreingestellten Himmelskörpern wählen oder dir eigene erstellen. 
                    Mit "`+"' fügst du ein Objekt hinzu, mit "`{\sffamily x}"' löschst du es.\\
                    Über das Feld "`Load configuration"' kannst du voreingestellte Sets an Himmelskörpern auswählen. Dabei werden alle aktuell erstellten Objekte gelöscht, bevor die neuen hinzugefügt werden.
                \end{itemize}
            \item Rechter Bereich:  
                Die Achsen sind in der Einheit Kilometer angegeben. 
                Du kannst in diesem Fenster mit dem Mausrad zoomen und durch klicken und ziehen die Karte bewegen.
                Oben links werden die vergangenen Tage angezeigt.
            \item Funktion der oberen Leiste:
                \begin{itemize}
                    \item \textbf{Start simulation/Restart simulation:} Startet die Berechnung mit den Numerik- und Objekteinstellungen, die im linken Bereich gesetzt wurden.
                    Wenn diese geändert wurden, muss dieser Button erneut gedrückt werden, um die Berechnung zu starten.
                    \item \textbf{Reset settings:} Setzt alle Einstellungen auf die Starteinstellungen zurück.
                    \item \textbf{Pause/Continue:} Pausiert die Anzeige/ Lässt die Anzeige weiterlaufen
                    \item \textbf{Updates per second:} Gibt an, wie oft pro Sekunde die Anzeigen aktualisiert werden. 
                            Wenn du die Zahl verringerst, kannst du schnelle Veränderungen gut beobachten.
                    \item \textbf{Days per update:} Gibt an um wie viele Tage bei einer Aktualisierung der Anzeige gesprungen wird.
                            Wenn du die Zahl erhöhst, laufen langsame Vorgänge schneller ab.
                \end{itemize}
        \end{itemize}
        
        \vspace{1cm}
        {\large\textbf{\sffamily Bei Problemen und Fragen scheut euch nicht, bei uns nachzufragen!}}\\[3cm]
        \vfill
        \begin{center}
        \noindent\textbf{Besuche gerne unsere \href{https://www.physik.uni-kl.de/rethfeld/}{Homepage}, wenn dir dieser Workshop gefällt: \\
        Dort findest du das Programm zum selbst installieren.\\}
        \includegraphics*[width=.2\textwidth]{qr_rethfeld.png}\\
        \vspace{2cm}
        \textbf{Dir gefällt es hier sehr gut:\\ Dann schreib dich ein für ein Physik-Studium an der RPTU.\\
        \url{https://physik.rptu.de/studium}}\\
        \includegraphics*[width=.2\textwidth]{qr_studium.png}
        \end{center}
        \newpage
    
\begin{center}
    {\huge\textbf{\sffamily Physik im Weltall:\\ Computersimulation von Planetenbewegungen}}\\[.75cm]
        {\Large\textbf{\sffamily Arbeitshinweise}}
\end{center}

        Kommt gerne auf uns zu, wenn ihr über bestimmte Aufgaben mit uns diskutieren möchtet.\\
        \subsection*{Teste die Numerik}
        Unser Programm erlaubt die Auswahl von zwei verschiedenen Lösungsverfahren:\\
        Dem sehr simplen Euler Verfahren und dem komplexeren Runge Kutta 4 (RK4).\\

        \noindent	Wähle Konfiguration "`sun\_earth"', Verfahren "`Euler"' und einen Zeitschritt von $3$ Tagen.
            \begin{itemize}
                \item Beobachte die Bahn der Erde. Was passiert, wenn du den Zeitschritt kleiner machst oder die Methode auf "`RK4"' umstellst? 
                \item Kannst du erklären, warum die Bahn bei den gegebenen Einstellungen so aussieht?
                \item Welche Vor- und Nachteile haben die verschiedenen Zeitschritte und Lösungsverfahren?
            \end{itemize}
        \subsection*{Zwei-Körper-Problem}
        Die Bewegung von zwei Massen, die sich unter dem Einfluss einer zentralen Kraft befinden (z.\,B. Gravitationskraft), kann mathematisch exakt vorhergesagt werden.
        Kleine Änderungen an den Startpositionen und Startgeschwindigkeiten der Massen führen zu kleinen Änderungen der Bewegungskurve.
        Solche Bahnen werden stabil genannt\footnote{In der Simulation kann es aufgrund von numerischen Fehlern zu unerwarteten Bewegungen kommen.}.\\

        \noindent Mit den Konfigurationen "`sun\_earth"', "`two\_suns"' und "`sun\_catches\_planet"' kannst du die Dynamik von verschiedenen Zwei-Körper-Problemen untersuchen.\\
        Konfiguration "`sun\_earth"':
        \begin{itemize}
            \item Was verändert sich, wenn du die Masse der Erde um 6 Zehnerpotenzen nach oben oder unten veränderst? Warum verhält sich das System jetzt so?
            \item Warum geben wir immer eine Startgeschwindigkeit an? Was würde passieren, wenn diese Null ist?\footnote{Wenn sich zwei Himmelskörper in unserer Simulation zu nahe kommen, werden diese aufgrund der numerischen Fehler zu stark beschleunigt.}
        \end{itemize}
        Konfigurationen "`two\_suns"' und "`sun\_catches\_planet"':
        \begin{itemize}
            \item Warum bewegen sich die Massen schneller, wenn sie näher beieinander sind?
            Kennst du ein Gesetz, welches dieses Verhalten beschreibt?
            \item Bleiben die Bahnen geschlossen, auch wenn du die Startparameter leicht veränderst?
        \end{itemize}
        \vfill\hfill
        Weiter geht es auf der nächsten Seite\dots\\[1cm]
        \phantom{A}
        \newpage
        \subsection*{Drei-Körper-Problem}
        Im Gegensatz zum Zwei-Körper-Problem ist bereits ein System von drei oder mehr Körpern nur noch in Ausnahmefällen exakt beschreibbar.
        In diesen Fällen sind numerische Methoden häufig die einzige Möglichkeit um neue physikalische Erkenntnisse zu erhalten.
        Drei-Körper-Probleme sind in der Regel instabil.\\
        
        \noindent Das bedeutet, dass kleine Variationen der Startwerte zu starken Veränderungen der Bahnkurve führen können.
        
        Ein stabiles Drei-Körper-Problem ist in der Konfiguration "`three\_body"' gespeichert.
        \begin{itemize}
            \item Verändere die Parameter der Himmelskörper und beobachte, wie sich die Bahnen verändern.
            \item Füge zu den Konfigurationen der Zwei-Körper-Probleme weitere Himmelskörper hinzu, die die Bahnen der Himmelskörper stark verändern. 
            \item Lade die Konfiguration "`sun\_earth\_moon"'. Wie könntest du aus der Geschwindigkeit des Mondes im Graph unten links die Geschwindigkeit des Mondes in Relation zur Erde erhalten? 
            (Tipp: Verringere den Planetenradius auf $8\cdot10^4\,$km und stelle ein, dass das Koordinatensystem der Erde folgt.)
        \end{itemize}
        \subsection*{Sonnensystem}
        Lade die Konfiguration "`solar\_system"'.
        \begin{itemize}
            \item Beobachte die Bewegung der Sonne. Was beeinflusst diese hauptsächlich? 
        \end{itemize}
        Einige Himmelskörper im äußeren Bereich unseres Sonnensystems weisen ungewöhnliche Bahnen auf.
        Bisher gibt es keine bestätigte Begründung für diese Bahnen. 
        Manche Experten vermuten einen neunten Planeten, der bisher noch nicht gefunden wurde.
        Neuste Berechnungen liefern nun eine andere mögliche Begründung:
        Vor Milliarden von Jahren könnte eine Sonne an unserem Sonnensystem vorbeigeflogen sein, die die Himmelskörper auf diese Bahnen gebracht hat\footnote{Weitere Informationen findest du über den \href{https://www.fr.de/wissen/simulationen-aus-deutschland-zeigen-planet-koennte-etwas-ganz-anderes-sein-zr-93300641.html}{QR-Code}.}.

        Die Konfiguration "`solar\_system\_disturbed"' erstellt ein solches Sonnensystem mit zweiter Sonne.
        Im Gegensatz zu der Sonne, die unser Sonnensystem vor Milliarden von Jahren besucht haben könnte, beeinflusst diese Sonne das Sonnensystem deutlich stärker.
        \begin{itemize}
            \item Beobachte die Bewegung der Himmelskörper.
            \item Warum verlassen manche Himmelskörper das Sonnensystem?
            \item Welche Planeten sind unbeeinflusst von den Geschehnissen und warum?
        \end{itemize}

        \subsection*{Weitere Planetensysteme}
        \begin{itemize}
        \item Lass deiner Kreativität freien Lauf! Erstelle neue Himmelskörper oder ändere die Daten der bereits existierenden Himmelskörper ab.
        \end{itemize}
        \vfill\hfill
        \includegraphics[width=.2\textwidth]{qr.png}

\end{document}
